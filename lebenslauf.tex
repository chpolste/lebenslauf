% CV template for the ConTeXt document processing system. Allows specifiction
% of multiple configurations and/or langugages in one document and control the
% processing via the command line.

% Personal details at the start of the CV
\definelabelclass[information][0]
% Headings for the categories of CV entries
\definelabelclass[category][0]
% Auto-generated text (internal)
\definelabelclass[multilang][0]

% Select language via command-line argument
\mainlanguage[\getdocumentargument{lang}]

% Language agnostic
\setupmultilangtext[footersep={$\coloncolon$}]

% English
\setuplanguage[en][date={dd,month,year}]
\setupmultilangtext[en][%
    cv={Curriculum Vitae},%
    page=page,%
    pageof=of,%
    since=since,%
    until=until,%
    name=Name,%
    address=Address,%
    phone=Phone,%
    email=E-mail,%
    contact=Contact,%
    birth=Born,%
    nationality=Nationality%
]

% Deutsch
\setuplanguage[de][date={dd,{.},mm,{.},year}]
\setupmultilangtext[de][%
    cv=Lebenslauf,%
    page=Seite,%
    pageof=von,%
    since=seit,%
    until=bis,%
    name=Name,%
    address=Adresse,%
    phone=Telefon,%
    email=E-Mail,%
    contact=Kontakt,%
    birth=Geboren,%
    nationality=Nationalität%
]

% Text in left column is colored dark grey by default
\setupcolors[state=start]
\definecolor[leftcol][r=.3,g=.3,b=.3]

% Document setup
\setuppapersize[A4]
\setuplayout[backspace=25mm,width=160mm,header=0mm,footer=6mm,footerdistance=8mm]
\setupinteraction[state=start,color=black,contrastcolor=black,style=,focus=standard]

% Text setup
\setupbodyfont[11pt,sans]
\definebodyfontenvironment[11pt][a=1.8em,b=1.3em,c=0.85em]
\setupinterlinespace[line=1.25em]
\setupwhitespace[medium]

% Footer setup: border at the top, name on the right, pagenumber on the right
\setupbackgrounds[footer][text][topframe=on]
\setupfootertexts%
    [\multilangtext{cv} \multilangtext{footersep} \informationtext{name}]%
    [\multilangtext{page} {\pagenumber} \multilangtext{pageof} {\totalnumberofpages}]

% The macro \testpage is used to avoid pagebreaks between category headings and
% entries. The parameter should allow at least one entry beneath a heading, but
% might have to tweaked for certain layouts.
\def\testpageparameter{10}

% Typesetting the CV: general table layout
\setupTABLE[split=yes,frame=off,loffset=0pt,roffset=0pt,boffset=4pt,toffset=1pt,align=flushleft]
\setupTABLE[c][1][width=40mm,style=leftcol,roffset=2pt]
\setupTABLE[c][2][width=120mm]

% CV Title
\define\cv{\framed[frame=off,offset=0pt]{\tfa\multilangtext{cv}}\blank[25pt]}

% Personal information: setup content \setupinformationtext first, then insert
% the desired fields in an information environment with \information[key].
% A set of default keys is specified, new keys can be added by setting the
% corresponding multilangtext values for each language.
\def\startinformation#1\stopinformation{\bTABLE #1 \eTABLE}
\def\information[#1]{
    \bTR
        \bTD \multilangtext{#1} \eTD
        \bTD \informationtext{#1} \eTD
    \eTR
}

% CV entries are organized into categories. The category heading is hidden if
% there is no visible entry, by rendering the entries into a box first and then
% checking if there is anything in the box. The emptiness test is adapted from
% https://tex.stackexchange.com/questions/53068/, but uses \hd instead of \wd
% since it works better with ConTeXt's modes for some reason.
\def\startcategory[#1]#2\stopcategory{
    \setbox0=\hbox{\bTABLE#2\eTABLE\unskip}\ifdim\hd0=0pt
        \empty
    \else
        \testpage[\testpageparameter]% make sure there is enough room to fit heading and at least one entry
        \bTABLE
            \bTR
                \bTD[nc=2,toffset=20pt,boffset=8pt,style=no] {\tfb\categorytext{#1}} \eTD
            \eTR
            #2 % rows of entries
        \eTABLE
    \fi
}

% An entry is associated with a language and three pieces of content: a text
% for the left column, a summary for the right column and (optionally) a more
% detailed description of the entry under the summary, typeset in a smaller
% font: \entry[lang]{leftcol}{summary}{details}
% The language association is implemented using modes, which is possible
% because ConTeXt automatically creates a mode for the selected language.
\def\entry[#1]#2#3#4{
    \startmode[**#1]
        \bTR
            \bTD #2 \eTD
            \bTD #3 \ifx#4\empty \empty \else \break {\tfc #4} \fi \eTD
        \eTR
    \stopmode
}

% Specialized entry whose text in the left column denotes a period of time:
% \periodentry[lang]{from}{to}{title}{summary}{details}. Either from or to can
% be omitted and the template will automatically apapt the display of the time
% period in the left column.
\def\periodentry[#1]#2#3#4#5{
    \entry[#1]{
        \def\temptwo{#2}
        \def\tempthree{#3}
        \ifx\temptwo\empty
            \multilangtext{until} #3
        \else
            \ifx\tempthree\empty
                \multilangtext{since} #2
            \else
                #2 - #3
            \fi
        \fi
    }{#4}{#5}
}

% End the CV with the current date and place for a signature (either empty
% space if printed or via a picture of the signature (specified with the
% optional argument).
\def\dateandsignature#1{
    \def\temp{#1}
    \bTABLE[split=no,frame=off,offset=0mm]
        \bTR
            \bTD[width=160mm,height=25mm,align=low,boffset=-1mm] \ifx\temp\empty \empty \else \externalfigure[#1][height=12mm] \fi \eTD
        \eTR
        \bTR
            \bTD[style=black,align=high] {\tfc\informationtext{name}, \currentdate} \eTD
        \eTR
    \eTABLE
}

