% CV preset for the ConTeXt document processing system. Allows specifiction of
% multiple configurations and/or langugages in one document and control the
% processing via the command line.

% Personal details in the header of the CV
\definelabelclass[information][0]
% Headings for the categories of CV entries
\definelabelclass[category][0]
% Auto-generated text (internal)
\definelabelclass[multilang][0]

% Select language via command-line argument
\mainlanguage[\getdocumentargument{lang}]

% Language agnostic
\setupmultilangtext[to=--]

% English
\setuplanguage[en][date={dd,month,year}]
\setupmultilangtext[en][%
    contact=Contact,%
    since=since,%
    until=until,%
]

% Deutsch
\setuplanguage[de][date={dd,{.},mm,{.},year}]
\setupmultilangtext[de][%
    contact=Kontakt,%
    since=seit,%
    until=bis,%
]

% Text in left column is colored dark grey by default
\setupcolors[state=start]

% Clickable hyperlinks in the pdf
\setupinteraction[state=start,color=black,contrastcolor=black,style=,focus=standard]

% Text setup
\setupbodyfont[11pt,heros,ss]
\setupinterlinespace[line=1.3em]
\setupwhitespace[medium]

% Headings
\setuphead[subject][style=\tfb,textcommand=\groupedcommand{}{\filler[SectionRightRule]}]
% https://tex.stackexchange.com/questions/497352
\definefiller[SectionRule][alternative=rule,color=header-fg-dark,height=5pt,depth=-4pt]
\definefiller[SectionRightRule][SectionRule][left=\quad,right=\zerowidthnobreakspace] 

% The macro \testpage is used to avoid pagebreaks between category headings and
% entries. The parameter should allow at least one entry beneath a heading, but
% might have to tweaked for certain layouts.
\def\testpageparameter{10}

% Global document setup
\setuppapersize[A4]
\setuplayout[
    backspace=10mm,
    topspace=10mm,
    width=190mm,
    height=277mm,
    margin=0mm,
    footer=0mm,
    header=12mm,
    headerdistance=12mm,
]
% Header with name and contact information on every page
\setupheadertexts[\directsetup{headertext}]
\startsetups [headertext]
    \hfill
    \setupcolors[textcolor=header-fg-dark]
    \bTABLE[option=stretch,frame=off]
        \setupTABLE[c][1][align=flushleft,width=140mm]
        \setupTABLE[c][2][align=flushright,width=50mm]
        \bTR
            \bTD \vfill \startcolor[header-fg-light] \tfb \bf \informationtext{name} \stopcolor \eTD
            \bTD \informationtext{phone}\break\informationtext{email} \eTD
        \eTR
     \eTABLE
\stopsetups
% Add a bit of color to the header
\definelayer[bg][x=0mm,y=0mm,width=\paperwidth, height=\paperheight,state=continue]
\setlayer[bg][]{%
    \framed[
        width=210mm,
        height=25mm,
        frame=off,
        bottomframe=on,
        framecolor=highlight,
        rulethickness=2pt,
        background=color,
        backgroundcolor=header-bg,
    ]{}
}

% First page setup
\definelayout[1][
    header=25mm,
]
% Use page marker to make the first page special:
% https://wiki.contextgarden.net/FO_Page_Layout#Changing_Headers_and_Footers_for_the_Firstpage
\setuphead[part][header=coverheader,placehead=empty] 
% Show more information on the first page
\definetext[coverheader][header][\directsetup{coverheadertext}]
\startsetups [coverheadertext]
    \hfill
    \setupcolors[textcolor=header-fg-dark]
    \bTABLE[frame=off]
        \setupTABLE[c][1][align=flushleft,width=140mm] % 95mm with picture TODO
        \setupTABLE[c][2][align=flushleft,width=50mm,leftframe=on,framecolor=highlight,rulethickness=1pt,loffset=3mm]
        \bTR
            \bTD[nr=2,boffset=4mm] \vfill \startcolor[header-fg-light] \tfd \bf \informationtext{name} \stopcolor \eTD
            \bTD \startcolor[highlight] \multilangtext{contact} \stopcolor \eTD
        \eTR
        \bTR
            \bTD \informationtext{address} \eTD
        \eTR
        \bTR
            \bTD \informationtext{headera}\break\informationtext{headerb} \eTD
            \bTD \informationtext{phone}\break\informationtext{email} \eTD
        \eTR
     \eTABLE
\stopsetups
% Larger box at the top of the first page
\definelayer[coverbg][x=0mm,y=0mm,width=\paperwidth,height=\paperheight]
\setlayer[coverbg][]{%
    \framed[
        width=210mm,
        height=45mm,
        frame=off,
        bottomframe=on,
        framecolor=highlight,
        rulethickness=2pt,
        background=color,
        backgroundcolor=header-bg,
    ]{}
}

\setupbackgrounds[page][][background={coverbg,bg}]


% CV entries are organized into categories. The category heading is hidden if
% there is no visible entry, by rendering the entries into a box first and then
% checking if there is anything in the box. The emptiness test is from
% https://tex.stackexchange.com/questions/53068/.
\def\startcategory[#1]#2\stopcategory{
    \setbox0=\hbox{#2\unskip}\ifdim\wd0=0pt
        \empty
    \else
        \testpage[\testpageparameter]% make sure there is enough room to fit heading and at least one entry
        \startsubject[title=\categorytext{#1}]
            \bTABLE
                \setupTABLE[split=yes,frame=off,offset=0mm,boffset=2mm,align=flushleft]
                \setupTABLE[c][1][width=45mm]
                \setupTABLE[c][2][width=145mm]
                \bTR[boffset=0mm] \bTD\eTD \bTD\eTD \eTR % ensure that there are always 2 columns
                #2
            \eTABLE
        \stopsubject
    \fi
}

% The language association for all entries is implemented using modes, which is
% possible because ConTeXt automatically creates a mode for the selected
% language.

% \tableentry[lang]{leftcol}{headline}{details}
% A table entry has three pieces of content: a text for the left column,
% a headline for the right column and (optionally) a more detailed description
% of the entry under the headline, typeset in a smaller font.
\def\tableentry[#1]#2#3#4{
    \startmode[**#1]
        \bTR
            \bTD[style=highlight,roffset=3mm] #2 \eTD
            \bTD #3 \ifx#4\empty \empty \else \break { \tfx #4} \fi \eTD
        \eTR
    \stopmode
}

% \periodentry[lang]{from}{to}{title}{headline}{details}
% Specialized entry whose text in the left column denotes a period of time.
% Either from or to can be left empty and the template will automatically apapt
% the display of the time period in the left column.
\def\periodentry[#1]#2#3#4#5{
    \tableentry[#1]{
        \def\temptwo{#2}
        \def\tempthree{#3}
        \ifx\temptwo\empty
            \multilangtext{until} #3
        \else
            \ifx\tempthree\empty
                \multilangtext{since} #2
            \else
                #2 \multilangtext{to} #3
            \fi
        \fi
    }{#4}{#5}
}

% \listentry[lang]{headline}{details}
\def\listentry[#1]#2#3{
    \startmode[**#1]
        \bTR
            \bTD[nc=2]
                \startitemize[1,joinedup][color=highlight]
                    \item #2 \ifx#3\empty \empty \else \break { \tfx #3} \fi
                \stopitemize
            \eTD
        \eTR
    \stopmode
}

